\newglossaryentry{Quidditch}
{
  name=Quidditch,
  description={est un sport fictif issu de la saga
Harry Potter
créée par
J. K. Rowling.
Chaque équipe possède sept joueurs chevauchant des
balais volants. L'objectif étant de
marquer plus de points que l'adversaire en
marquant un maximum de buts et en
attrapant une balle magique, le
Vif d'or. Un match peut durer des mois et comporte des
risques mortels pour les joueurs}
}
\newglossaryentry{manager}
{
  name=manager,
  description={est l'utilisateur du programme, en charge de la gestion d'un club de Quidditch
  et de ses matchs}
}
\newglossaryentry{joueur}
{
  name=joueur,
  description={est un membre d'une équipe de Quidditch}
}
\newglossaryentry{match}
{
  name=match,
  description={est un affrontement de deux équipes de Quidditch sur un terrain}
}
\newglossaryentry{club}
{
  name=club,
  description={est un ensemble reprenant une équipe de Quidditch et son infrastructure}
}
\newglossaryentry{partie}
{
  name=partie,
  description={désigne la session de jeu que s'est créé l'utilisateur/Manager et qu'il peut gérer à sa guise. Il y a une partie par utilisateur}
}
\newglossaryentry{parcelle}
{
  name=parcelle,
  description={espace alloué à l'utilisateur où se trouvent ses bâtiments et son terrain de Quidditch}
}
\newglossaryentry{stade}
{
  name=stade,
  description={lieu sur l’espace alloué à un Manager où sont joués les matchs de Quidditch}
}
\newglossaryentry{terrain}
{
  name=terrain,
  description={espace à l'intérieur du stade où se déplacent les joueurs et les balles au cours d'un match}
}
\newglossaryentry{infirmerie}
{
  name=infirmerie,
  description={lieu où les joueurs du Managers sont soignés}
}
\newglossaryentry{centreentrainement}
{
  name=centre d’entrainement,
  description={lieu où le Manager peut faire s'entrainer ses joueurs}
}
\newglossaryentry{agencepublicite}
{
  name=agence de publicité,
  description={lieu où le Manager peut faire augmenter la popularité de ses joueurs}
}
\newglossaryentry{buvette}
{
  name=buvette,
  description={lieu où les supporters de l'équipe en déplacement pour un match qui ne se rendent pas sur place peuvent assister à une retransmission du match. Les consommations sont payantes et les bénéfices sont reversés au Manager}
}
\newglossaryentry{fanshop}
{
  name=fanshop,
  description={lieu où les supporters de l'équipe qui reçoit une équipe adverse pour un match peuvent acheter des produits dérivés}
}
\newglossaryentry{magasinbalais}
{
  name=magasin de balais,
  description={lieu où le Manager peut acheter/vendre des balais}
}
\newglossaryentry{centrerecrutement}
{
  name=centre de recrutement,
  description={lieu où le Manager peut acheter/vendre des joueurs}
}
\newglossaryentry{calendrier}
{
  name=calendrier,
  description={ensemble de rappels fixés dans le temps, le Manager est régulièrement rappelé à certaines actions (comme jouer un match) sur base de ce calendrier}
}

\newglossaryentry{evenement}
{
  name=événement,
  description={Action inscrite dans un calendrier qui rend possibles certains actions (jouer un match, sélectionner un joueur auparavant bloqué,...)}
}
\newglossaryentry{attrapeur}
{
  name=attrapeur,
  description={joueur dont la fonction au sein de l’équipe en cours de match est d’attraper le Vif d’Or}
}
\newglossaryentry{poursuiveur}
{
  name=poursuiveur,
  description={au nombre de 3 par équipe au cours d'un match, font des passes de Souafle entre eux et tentent de marquer des buts. Quand c'est l'équipe adverse qui a le souafle, ils essaient de le récupérer}
}
\newglossaryentry{batteur}
{
  name=batteur,
  description={au nombre de deux par équipe au cours d'un match. Un batteur frappe sur les Cognards et les envoie sur les joueurs adverses pour les blesser/déstabiliser}
}
\newglossaryentry{gardien}
{
  name=gardien,
  description={il n’y en a qu’un par équipe. Son rôle est d’empêcher les Poursuiveurs de l’équipe adverse de marquer des buts en interceptant leurs tirs}
}
\newglossaryentry{souafle}
{
  name=souafle,
  description={balle inerte unique manipulée pendant un match par les Poursuiveurs à travers des passes et des tentatives de tirs dans l’un des cercles d’or pour marquer un but}
}
\newglossaryentry{cognard}
{
  name=cognard,
  description={au nombre de deux. Plus petite que le Souafle. Balle magique qui essaie de frapper les joueurs pour les faire tomber de leur balais. Ces balles sont cognées par les batteurs pour les diriger vers les joueurs adverses}
}
\newglossaryentry{vifOr}
{
  name=vif d’Or,
  description={petite balle unique ayant sa propre volonté, qui se déplace aléatoirement et très rapidement dans les airs. Lorsqu’un Attrapeur attrape le Vif d’Or, la partie s’achève}
}
